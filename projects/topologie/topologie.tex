\documentclass[12pt,]{article}
\usepackage{lmodern}
\usepackage{amssymb,amsmath}
\usepackage{ifxetex,ifluatex}
\usepackage{fixltx2e} % provides \textsubscript
\ifnum 0\ifxetex 1\fi\ifluatex 1\fi=0 % if pdftex
  \usepackage[T1]{fontenc}
  \usepackage[utf8]{inputenc}
\else % if luatex or xelatex
  \ifxetex
    \usepackage{mathspec}
    \usepackage{xltxtra,xunicode}
  \else
    \usepackage{fontspec}
  \fi
  \defaultfontfeatures{Mapping=tex-text,Scale=MatchLowercase}
  \newcommand{\euro}{€}
    \setmainfont{"Palatino"}
    \setmonofont[Mapping=tex-ansi]{"Consolas"}
\fi
% use upquote if available, for straight quotes in verbatim environments
\IfFileExists{upquote.sty}{\usepackage{upquote}}{}
% use microtype if available
\IfFileExists{microtype.sty}{%
\usepackage{microtype}
\UseMicrotypeSet[protrusion]{basicmath} % disable protrusion for tt fonts
}{}
\ifxetex
  \usepackage[setpagesize=false, % page size defined by xetex
              unicode=false, % unicode breaks when used with xetex
              xetex]{hyperref}
\else
  \usepackage[unicode=true]{hyperref}
\fi
\hypersetup{breaklinks=true,
            bookmarks=true,
            pdfauthor={},
            pdftitle={},
            colorlinks=true,
            citecolor=blue,
            urlcolor=blue,
            linkcolor=magenta,
            pdfborder={0 0 0}}
\urlstyle{same}  % don't use monospace font for urls
\usepackage{fancyhdr}
\pagestyle{fancy}
\pagenumbering{arabic}
\lhead{\itshape }
\chead{}
\rhead{\itshape{\nouppercase{\leftmark}}}
\lfoot{v 1.0}
\cfoot{}
\rfoot{\thepage}
\usepackage{graphicx,grffile}
\makeatletter
\def\maxwidth{\ifdim\Gin@nat@width>\linewidth\linewidth\else\Gin@nat@width\fi}
\def\maxheight{\ifdim\Gin@nat@height>\textheight\textheight\else\Gin@nat@height\fi}
\makeatother
% Scale images if necessary, so that they will not overflow the page
% margins by default, and it is still possible to overwrite the defaults
% using explicit options in \includegraphics[width, height, ...]{}
\setkeys{Gin}{width=\maxwidth,height=\maxheight,keepaspectratio}
\setlength{\parindent}{0pt}
\setlength{\parskip}{6pt plus 2pt minus 1pt}
\setlength{\emergencystretch}{3em}  % prevent overfull lines
\providecommand{\tightlist}{%
  \setlength{\itemsep}{0pt}\setlength{\parskip}{0pt}}
\setcounter{secnumdepth}{5}

\date{}

% Redefines (sub)paragraphs to behave more like sections
\ifx\paragraph\undefined\else
\let\oldparagraph\paragraph
\renewcommand{\paragraph}[1]{\oldparagraph{#1}\mbox{}}
\fi
\ifx\subparagraph\undefined\else
\let\oldsubparagraph\subparagraph
\renewcommand{\subparagraph}[1]{\oldsubparagraph{#1}\mbox{}}
\fi

\begin{document}

{
\hypersetup{linkcolor=black}
\setcounter{tocdepth}{3}
\tableofcontents
}
\subsection{Topologie}\label{topologie}

\subsubsection{Def: Offene Mengen}\label{def-offene-mengen}

Sei \(X\) eine Menge und
\(\mathfrak{T} \subset \mathcal{P}\left(X\right)\) eine Menge von
Teilmengen von \(X\). \((X,\mathfrak{T})\) heißt \emph{topologischer
Raum}, wenn gilt:

\begin{enumerate}
\tightlist
\item
  \(X \in \mathfrak{T}\) und \(\emptyset \in \mathfrak{T}\).
\item
  Sei \(I\) eine beliebige Indexmenge und für alle \(i \in I\) gilt:
  \(T_i \in \mathfrak{T}\), dann ist auch
  \(\displaystyle\bigcup_{i \in I} T_i \in \mathfrak{T}\).
\item
  Für endliche Indexmengen \(I\) und \(T_i \in \mathfrak{T}\) gilt
  \(\displaystyle\bigcap_{i \in I} T_i \in \mathfrak{T}\).
\end{enumerate}

\subsubsection{Offene Mengen}\label{offene-mengen}

Ist \((X,\mathfrak{T})\) ein topologischer Raum, so nennt man die
Elemente von \(\mathfrak{T}\) offene Mengen.

\subsubsection{Ursprung des Begriffs}\label{ursprung-des-begriffs}

Die \textbf{Topologie} ( \emph{} ‚Ort`, ‚Platz` und \url{-logie}) ist
ein fundamentales \href{Teilgebiete_der_Mathematik}{Teilgebiet der
Mathematik}. Sie beschäftigt sich mit den Eigenschaften mathematischer
Strukturen, die unter stetigen Verformungen erhalten bleiben, wobei der
Begriff der \href{Stetigkeit_(Topologie)}{Stetigkeit} durch die
Topologie in sehr allgemeiner Form definiert wird. Die Topologie ging
aus den Konzepten der \url{Geometrie} und \url{Mengenlehre} hervor.

\subsubsection{19.~Jahrhunderts}\label{jahrhunderts}

Gegen Ende des 19.~Jahrhunderts entstand die Topologie als eine
eigenständige Disziplin, die auf lateinisch \emph{} ‚Geometrie der Lage`
oder \emph{} (Griechisch-Latein für ‚Analysieren des Ortes`) genannt
wurde.

\subsubsection{Heutige Bedeutung}\label{heutige-bedeutung}

Seit Jahrzehnten ist die Topologie als Grundlagendisziplin anerkannt.
Dementsprechend kann sie neben der \url{Algebra} als zweiter
Stützpfeiler für eine große Anzahl anderer Felder der Mathematik
angesehen werden. Sie ist besonders wichtig für die \url{Geometrie}, die
\url{Analysis}, die \url{Funktionalanalysis} und die Theorie der
\href{Lie-Gruppe}{Lie-Gruppen}. Ihrerseits hat sie auch die
\url{Mengenlehre} und \url{Kategorientheorie} befruchtet.

\subsubsection{Beispiel für stetige
Deformation}\label{beispiel-fuxfcr-stetige-deformation}

Tasse und Torus sind zueinander homöomorph

\begin{figure}[htbp]
\centering
\includegraphics{./images/mugmorph.gif}
\caption{Stetige Deformation}
\end{figure}

\subsubsection{Homöomorphismus}\label{homuxf6omorphismus}

Ein Homöomorphismus ist eine direkte Abbildung zwischen den Punkten der
Tasse und des Torus, die Zwischenstufen im zeitlichen Verlauf dienen nur
der Illustration der Stetigkeit dieser Abbildung.

\subsubsection{Abbildung zwischen topologischen
Räumen}\label{abbildung-zwischen-topologischen-ruxe4umen}

Der grundlegende Begriff der Topologie ist der des
\href{Topologischer_Raum}{topologischen Raums}, welcher eine
weitreichende Abstraktion der Vorstellung von „Nähe`` darstellt und
damit weitreichende Verallgemeinerungen mathematischer Konzepte wie
\href{Stetige_Funktion}{Stetigkeit} und
\href{Grenzwert_(Folge)}{Grenzwert} erlaubt.

\subsubsection{Topologische
Eigenschaften}\label{topologische-eigenschaften}

Viele mathematische Strukturen lassen sich als topologische Räume
auffassen. \emph{Topologische Eigenschaften} einer Struktur werden
solche genannt, die nur von der Struktur des zugrundeliegenden
topologischen Raumes abhängen.

\subsubsection{Nachbarschaftserhaltende
Abbildungen}\label{nachbarschaftserhaltende-abbildungen}

\href{Homöomorphismus}{Homöomorphismen} sind nachbarschaftserhaltende
Abbildungen, die durch „Verformungen`` nicht verändert werden. Dazu
gehört in anschaulichen Fällen das

\begin{itemize}
\tightlist
\item
  Dehnen, Stauchen,
\item
  Verbiegen, Verzerren und
\item
  Verdrillen einer geometrischen Figur.
\end{itemize}

\subsubsection{Beispiele für homöomorphe
Objekte}\label{beispiele-fuxfcr-homuxf6omorphe-objekte}

Zum Beispiel sind eine Kugel und ein Würfel aus Sicht der Topologie
nicht zu unterscheiden; sie sind homöomorph. Ebenso sind ein Torus und
eine einhenkelige \url{Tasse} homöomorph.

\begin{figure}[htbp]
\centering
\includegraphics{./images/mugmorph.gif}
\caption{Stetige Deformation}
\end{figure}

\subsubsection{Gliederung der Topologie}\label{gliederung-der-topologie}

Die Topologie gliedert sich selbst in mehrere Teilgebiete. Hierzu zählen
die \href{algebraische_Topologie}{algebraische Topologie}, die
\href{geometrische_Topologie}{geometrische Topologie} sowie die
\href{Topologische_Graphentheorie}{topologische Graphen-} und die
\url{Knotentheorie}.

\subsubsection{Zentrale Probleme}\label{zentrale-probleme}

Ein zentrales Problem dieser Disziplinen ist der Versuch, Verfahren zu
entwickeln, zu beweisen, dass zwei Räume nicht homöomorph sind, oder
allgemeiner, dass stetige Abbildungen mit bestimmten Eigenschaften nicht
existieren. Die \textbf{mengentheoretische Topologie} kann hierbei als
Grundlage für all diese Teildisziplinen angesehen werden. In dieser
werden insbesondere auch topologische Räume betrachtet, deren
Eigenschaften sich im Allgemeinen besonders weit von denen geometrischer
Figuren unterscheiden.

\subsection{Geschichte}\label{geschichte}

\subsubsection{Entstehung des Beriffs}\label{entstehung-des-beriffs}

Der Begriff „Topologie`` findet sich erstmals um 1840 bei
\href{Johann_Benedict_Listing}{Johann Benedict Listing}; die ältere
Bezeichnung \emph{} (etwa ‚Lageuntersuchung`) blieb aber lange üblich,
mit einem Bedeutungsschwerpunkt jenseits der neueren,
„mengentheoretischen`` Topologie.

\subsubsection{Königsberger
Brückenproblem}\label{kuxf6nigsberger-bruxfcckenproblem}

Die Lösung des
\href{Königsberger_Brückenproblem}{Sieben-Brücken-Problems von
Königsberg} durch \href{Leonhard_Euler}{Leonhard Euler} im Jahr 1736
gilt als die erste topologische und zugleich als die erste
\href{Graphentheorie}{graphentheoretische} Arbeit in der Geschichte der
Mathematik.\footnote{}\footnote{}

\subsubsection{Zusammenhang: Ecken, Kanten,
Flächen}\label{zusammenhang-ecken-kanten-fluxe4chen}

Ein anderer Beitrag Eulers zur sogenannten \emph{Analysis situs} ist der
nach ihm benannte \url{Polyedersatz} von 1750. Bezeichnet man mit \(e\)
die Anzahl der Ecken, mit \(k\) die der Kanten und mit \(f\) die der
\href{Fläche_(Mathematik)}{Flächen} eines \href{Polyeder}{Polyeders}
(der noch zu präzisierenden Bedingungen genügt), so gilt
\(e - k + f = 2\). Erst im Jahr 1860 wurde durch eine von
\href{Gottfried_Wilhelm_Leibniz}{Gottfried Wilhelm Leibniz} angefertigte
Abschrift eines verlorenen Manuskriptes von \href{René_Descartes}{René
Descartes} bekannt, dass dieser die Formel bereits gekannt
hatte.\footnote{Christoph J. Scriba, Peter Schreiber: \emph{5000 Jahre
  Geometrie: Geschichte, Kulturen, Menschen (Vom Zählstein zum
  Computer).} Springer, Berlin, Heidelberg, New York, ISBN
  3-540-67924-3, S. 451.}

\subsubsection{Metrische Räume als
Spezialfall}\label{metrische-ruxe4ume-als-spezialfall}

\href{Maurice_Fréchet}{Maurice Fréchet} führte 1906 den
\href{Metrischer_Raum}{metrischen Raum} ein.\footnote{}
\href{Georg_Cantor}{Georg Cantor} befasste sich mit den Eigenschaften
offener und abgeschlossener Intervalle, untersuchte Grenzprozesse, und
begründete dabei zugleich die moderne Topologie und die
\url{Mengentheorie}.\footnote{} Die Topologie ist der erste Zweig der
Mathematik, der konsequent mengentheoretisch formuliert wurde -- und gab
dabei umgekehrt Anstöße zur Ausformung der Mengentheorie.

\subsubsection{Geschichte der
Definition}\label{geschichte-der-definition}

\paragraph{1914}\label{section}

Eine Definition des \emph{topologischen Raumes} wurde als erstes von
\href{Felix_Hausdorff}{Felix Hausdorff}\footnote{Felix Hausdorff:
  \emph{\href{Grundzüge_der_Mengenlehre}{Grundzüge der Mengenlehre}},
  1914, S 213} im Jahre 1914 aufgestellt. Nach heutigem Sprachgebrauch
definierte er dort eine offene \href{Basis_(Topologie)}{Umgebungsbasis},
nicht jedoch eine Topologie.

\paragraph{1924}\label{section-1}

Der Begriff der Topologie wurde erst durch
\href{Kazimierz_Kuratowski}{Kazimierz Kuratowski}\footnote{\emph{Fund.
  Math.}, \textbf{3}, 1922} beziehungsweise
\href{Heinrich_Tietze}{Heinrich Tietze}\footnote{Math. Ann. 88, 1923} um
1922 eingeführt wurde. In dieser Form wurden die Axiome dann durch die
Lehrbücher von \href{Kazimierz_Kuratowski}{Kuratowski} (1933),
\href{Paul_Alexandroff}{Alexandroff}/\href{Heinz_Hopf}{Hopf} (1935),
\href{Nicolas_Bourbaki}{Bourbaki} (1940) und
\href{John_Leroy_Kelley}{Kelley} (1955) popularisiert.\footnote{Epple et
  al., Hausdorff GW II, 2002}

\subsubsection{Bezüge zu anderen
Disziplinen}\label{bezuxfcge-zu-anderen-disziplinen}

Es stellte sich heraus, dass sich viele mathematische Erkenntnisse auf
diese Begriffsbasis übertragen ließen. Es wurde beispielsweise erkannt,
dass zu einer festen Grundmenge unterschiedliche Metriken existieren,
die zur gleichen topologischen Struktur auf dieser Menge führten, aber
auch, dass verschiedene Topologien auf der gleichen Grundmenge möglich
sind.

\subsubsection{Mengentheoretische
Topologie}\label{mengentheoretische-topologie}

Die mengentheoretische Topologie entwickelte sich auf dieser Grundlage
zu einem eigenständigen Forschungsgebiet, das sich in gewisser Weise aus
der \url{Geometrie} ausgegliedert hat, beziehungsweise der
\url{Analysis} näher steht als der eigentlichen Geometrie.\footnote{Christoph
  J. Scriba, Peter Schreiber: \emph{5000 Jahre Geometrie: Geschichte,
  Kulturen, Menschen (Vom Zählstein zum Computer).} Springer, Berlin,
  Heidelberg, New York, ISBN 3-540-67924-3, S. 515.}

\subsubsection{Ziel der Topologie}\label{ziel-der-topologie}

Ein Ziel der Topologie ist die Entwicklung von
\href{Invariante_(Mathematik)}{Invarianten} von topologischen Räumen.
Mit diesen Invarianten können topologische Räume unterschieden werden.

\subsection{Grundbegriffe}\label{grundbegriffe}

\subsubsection{Topologischer Raum}\label{topologischer-raum}

 Die Topologie befasst sich mit Eigenschaften
\emph{\href{Topologischer_Raum}{topologischer Räume}}. Ein topologischer
Raum ist zunächst einmal eine Menge von \emph{Punkten}.

\subsubsection{Struktur durch abgeschlossene
Mengen}\label{struktur-durch-abgeschlossene-mengen}

Die Struktur des Raumes bestimmt sich dann dadurch, dass bestimmte
Teilmengen von Punkten als \emph{abgeschlossen} ausgezeichnet werden.
\href{Abgeschlossene_Menge}{Abgeschlossene Mengen} lassen sich als
Mengen von Punkten vorstellen, die ihren Rand enthalten, oder anders
ausgedrückt: Wann immer es Punkte der abgeschlossenen Menge gibt, die
beliebig nah an einen anderen Punkt heranreichen (einen
\emph{\href{Rand_(Topologie)}{Berührpunkt}}), ist auch dieser Punkt in
der abgeschlossenen Menge enthalten.

\subsubsection{Grundlegende Eigenschaften abgeschlossener
Mengen}\label{grundlegende-eigenschaften-abgeschlossener-mengen}

Man überlegt sich, welche grundlegenden Eigenschaften im Begriff der
abgeschlossenen Menge enthalten sein sollten und nennt dann, von
spezifischen Definitionen der Abgeschlossenheit, etwa aus der
\url{Analysis}, abstrahierend, jede mit diesen Bedingungen genügenden
abgeschlossenen Teilmengen versehene Menge einen topologischen Raum.

\subsubsection{Leere Menge:
abgeschlossen}\label{leere-menge-abgeschlossen}

Zunächst einmal sollte die leere Menge abgeschlossen sein, denn sie
enthält keinerlei Punkte, die andere \emph{berühren} könnten. Daraus
ergibt sich, dass das Komplement von \(\emptyset\) (also \(X\)) offen
sein muss.

\subsubsection{Gesamte Grundmenge:
abgeschlossen}\label{gesamte-grundmenge-abgeschlossen}

Ebenso sollte die Menge aller Punkte abgeschlossen sein, denn sie
enthält bereits alle möglichen Berührpunkte. Ist eine beliebige Menge
von abgeschlossenen Mengen gegeben, so soll der Schnitt, das heißt die
Menge der Punkte, die in allen diesen Mengen enthalten sind, ebenfalls
abgeschlossen sein, denn hätte der Schnitt Berührpunkte, die außerhalb
seiner liegen, so müsste bereits eine der zu schneidenden Mengen diesen
Berührpunkt nicht enthalten, und könnte nicht abgeschlossen sein.

\subsubsection{Vereinigung abgeschlossener
Mengen}\label{vereinigung-abgeschlossener-mengen}

Zudem soll die Vereinigung zweier (oder endlich vieler) abgeschlossener
Mengen wiederum abgeschlossen sein; bei der Vereinigung zweier
abgeschlossener Mengen kommen also keine Berührpunkte hinzu. Von der
Vereinigung unendlich vieler abgeschlossener Mengen dagegen fordert man
keine Abgeschlossenheit, denn diese könnten sich einem weiteren Punkte
„immer weiter nähern`` und somit berühren.

\subsubsection{Beliebiger Schnitt abgeschlossener
Mengen}\label{beliebiger-schnitt-abgeschlossener-mengen}

Ein beliebiger Schnitt abgeschlossener Menge sollte auch wieder
abgeschlossen sein.

\subsubsection{Definition über abgeschlossene
Mengen}\label{definition-uxfcber-abgeschlossene-mengen}

In Analogie zur Definition der Topologie über offene Mengen, kann man
einen topologischen Raum auch über die abgeschlossenen Mengen
definieren.

\subsubsection{Definition Topologie über abgeschlossene
Mengen}\label{definition-topologie-uxfcber-abgeschlossene-mengen}

Ein \emph{topologischer Raum} ist eine Menge von Punkten \(X\) versehen
mit einer Menge \(\mathfrak{S} \subset \mathcal{P}\left(X\right)\) von
Teilmengen von \(X\) (den abgeschlossenen Mengen;
\(\mathcal{P}\left(X\right)\) ist die \url{Potenzmenge} von \(X\)), die
folgenden Bedingungen genügt:

\begin{enumerate}
\tightlist
\item
  \(X \in \mathfrak{S}\) und \(\empty \in \mathfrak{S}\).
\item
  Sei \(I\) eine beliebige Indexmenge und für alle \(i \in I\) gilt und
  \(S_i \in \mathfrak{S}\), so ist auch
  \(\displaystyle\bigcap_{i \in I} S_i \in \mathfrak{S}\).
\item
  Für endliche Indexmengen \(I\) und \(S_i \in \subset \mathfrak{S}\)
  ist auch \(\displaystyle\bigcup_{i \in I} S_i \in \mathfrak{S}\).
\end{enumerate}

\subsubsection{Topologie als System offener
Mengen}\label{topologie-als-system-offener-mengen}

\textbf{Definition Topologie} \emph{(über offene Mengen)}: Ein
\emph{topologischer Raum} ist eine Menge von Punkten \(X\) versehen mit
einer Menge von Teilmengen
\(\mathfrak{T} \subset \mathcal{P}\left(X\right)\) (den offenen Mengen),
die folgenden Bedingungen genügt:

\begin{enumerate}
\tightlist
\item
  \(X \in \mathfrak{T}\) und \(\emptyset \in \mathfrak{T}\).
\item
  Sei \(I\) eine beliebige Indexmenge und für alle \(i \in I\) gilt:
  \(T_i \in \mathfrak{T}\), dann ist auch
  \(\displaystyle\bigcup_{in \in I} T_i \in \mathfrak{T}\).
\item
  Für endliche Indexmengen \(I\) und \(T_i \in \mathfrak{T}\) gilt auch
  \(\displaystyle\bigcap_{in \in I} T_i \in \mathfrak{T}\).
\end{enumerate}

Die Menge \(\mathfrak{T}\) der offenen Mengen wird auch als
\emph{Topologie} bezeichnet.

\subsubsection{Äquivalenz der
Definition}\label{uxe4quivalenz-der-definition}

Die Äquivalenz zur vorherigen Definition über abgeschlossene Mengen
folgt unmittelbar aus den \href{De_Morgan’sche_Gesetze}{De Morgan'schen
Gesetzen}. Ausgehend von abgeschlossenen beziehungsweise offenen Mengen
lassen sich zahlreiche topologische Begriffe definieren, etwa die der
\href{Umgebung_(Mathematik)}{Umgebung}, des Berührpunktes (welche zuvor
angesprochen wurden), der \href{Stetigkeit_(Topologie)}{Stetigkeit} und
der \href{Konvergenz_(Mathematik)}{Konvergenz}.

\subsubsection{Weitere Definitionen}\label{weitere-definitionen}

\subsubsection{Homöomorphismus}\label{homuxf6omorphismus-1}

 Ein \emph{Homöomorphismus} ist eine \href{Bijektion}{bijektive
Abbildung} zwischen zwei topologischen Räumen, sodass durch punktweise
Überführung der offenen Mengen auch eine Bijektion zwischen den
Topologien der beiden Räume zustande kommt, dabei muss jede offene Menge
auf eine offene Menge abgebildet werden. Zwei topologische Räume,
zwischen denen es einen Homöomorphismus gibt, werden als
\emph{homöomorph} bezeichnet. Homöomorphe Räume unterscheiden sich nicht
bezüglich aller topologischen Eigenschaften im engeren Sinne. Die
Homöomorphismen können als die \href{Isomorphismus}{Isomorphismen} in
der \href{Kategorientheorie}{Kategorie} der topologischen Räume
aufgefasst werden.

\subsubsection{Nicht auf topologische Räume bezogene
Begriffe}\label{nicht-auf-topologische-ruxe4ume-bezogene-begriffe}

Topologische Räume können mit Zusatzstrukturen ausgestattet werden,
beispielsweise untersucht man \href{Uniformer_Raum}{uniforme Räume},
\href{Metrischer_Raum}{metrische Räume},
\href{Topologische_Gruppe}{topologische Gruppen} oder
\href{Topologische_Algebra}{topologische Algebren}. Eigenschaften, die
auf solche Zusatzstrukturen zurückgreifen, sind nicht mehr unbedingt
unter Homöomorphismen erhalten, jedoch auch teils
Untersuchungsgegenstand verschiedener Teilgebiete der Topologie.

\subsubsection{Verallgemeinerungen Topologischer
Raum}\label{verallgemeinerungen-topologischer-raum}

Es existieren auch Verallgemeinerungen des Konzepts des topologischen
Raums: In der \href{Punktfreie_Topologie}{punktfreien Topologie}
betrachtet man an Stelle einer Menge von Punkten mit als offen
ausgezeichneten Mengen nur noch die Struktur der offenen Mengen als
\href{Verband_(Mathematik)}{Verband}.
\href{Konvergenzstruktur}{Konvergenzstrukturen} definieren, gegen welche
Werte jeder \href{Filter_(Mathematik)}{Filter} auf einer
zugrundeliegenden Menge von Punkten konvergiert. Unter dem Schlagwort
\emph{\href{Convenient_Topology}{Convenient Topology}} wird versucht,
Klassen von den topologischen oder uniformen Räumen ähnlichen Räumen zu
finden, die aber „angenehmere``
\href{Kategorientheorie}{kategorientheoretische} Eigenschaften
aufweisen.

\subsection{Teilgebiete der Topologie}\label{teilgebiete-der-topologie}

\subsubsection{Topologie
Differentialgeometrie}\label{topologie-differentialgeometrie}

Die moderne Topologie wird grob in die drei Teilgebiete
mengentheoretische Topologie, algebraische Topologie und geometrische
Topologie unterteilt. Außerdem gibt es noch die
\url{Differentialtopologie}. Dies ist die Grundlage der modernen
\url{Differentialgeometrie} und wird trotz der umfangreich verwendeten
topologischen Methoden meist als Teilgebiet der Differentialgeometrie
betrachtet.

\subsubsection{Mengentheoretische oder Allgemeine
Topologie}\label{mengentheoretische-oder-allgemeine-topologie}

Die mengentheoretische Topologie umfasst, wie auch die anderen
Teilgebiete der Topologie, das Studium topologischer Räume und der
stetigen Abbildungen zwischen ihnen.

\subsubsection{Topologie und Stetigkeit}\label{topologie-und-stetigkeit}

Insbesondere die für die \url{Analysis} fundamentalen Konzepte der
\href{Stetigkeit_(Topologie)}{Stetigkeit} und der
\href{Grenzwert_(Folge)}{Konvergenz} werden erst in der Terminologie der
mengentheoretischen Topologie vollständig transparent. Aber auch in
vielen anderen mathematischen Teilgebieten werden die Konzepte der
mengentheoretischen Topologie eingesetzt. Außerdem gibt es viele
\href{Satz_(Mathematik)}{mathematische Aussagen}, die nur in der
Terminologie der mengentheoretischen Topologie ihre natürliche und
allgemeinste Formulierung haben.

\subsubsection{Kompaktheit}\label{kompaktheit}

Beispielsweise ist die \href{Kompakter_Raum}{Kompaktheit eines Raums}
eine Abstraktion des \href{Satz_von_Heine-Borel}{Heine--Borel-Prinzips}.
In der allgemeinen Terminologie der mengentheoretischen Topologie gilt,
dass das Produkt zweier kompakter Räume wieder kompakt ist, was die
Aussage verallgemeinert, dass ein
\href{Abgeschlossene_Menge}{abgeschlossener} endlichdimensionaler Würfel
kompakt ist.

\subsubsection{Kompaktheit und
Extermstellen}\label{kompaktheit-und-extermstellen}

Außerdem gilt, dass eine stetige Funktion von einer kompakten Menge in
die reellen Zahlen beschränkt ist und ihr Maximum und Minimum annimmt.
Dies ist eine Verallgemeinerung des
\href{Satz_vom_Minimum_und_Maximum}{Satzes vom Minimum und
Maximum}.\footnote{}

\subsubsection{Trennungseigenschaften}\label{trennungseigenschaften}

Im Allgemeinen können topologische Räume viele etwa von der Topologie
der reellen Zahlen vertraute Eigenschaften verletzen, die jedoch in
üblichen Räumen häufig anzutreffen sind. Daher betrachtet man oftmals
topologische Räume, die gewissen
\emph{\href{Trennungseigenschaft}{Trennungseigenschaften}} genügen,
welche minimale Anforderungen für viele weitergehende Sätze darstellen
und tiefergehende Charakterisierungen der Struktur der Räume
ermöglichen. Die Kompaktheit ist ein anderes Beispiel für solche
„vorteilhaften`` Eigenschaften.

\subsubsection{Uniforme Räume, Metrik,
Vollständigkeit}\label{uniforme-ruxe4ume-metrik-vollstuxe4ndigkeit}

Zudem betrachtet man auch Räume, auf denen gewisse zusätzliche
Strukturen definiert sind, etwa \href{Uniformer_Raum}{uniforme Räume}
oder gar \href{topologische_Gruppe}{topologische Gruppen} und
\href{Metrischer_Raum}{metrische Räume}, welche durch ihre Struktur
zusätzliche Begrifflichkeiten wie die der
\href{Vollständiger_Raum}{Vollständigkeit} ermöglichen.

Ein anderer zentraler Begriff dieses Teilgebiets sind unterschiedliche
Konzepte von \href{Zusammenhängender_Raum}{Zusammenhang}.

\subsubsection{Algebraische Topologie}\label{algebraische-topologie}

Die algebraische Topologie (auch \emph{„kombinatorische Topologie``},
vor allem in älteren Publikationen) untersucht Fragestellungen zu
topologischen Räumen, indem die Probleme auf Fragestellungen in der
\url{Algebra} zurückgeführt werden. Innerhalb der Algebra sind diese
Fragen oftmals leichter zu beantworten. Ein zentrales Problem innerhalb
der Topologie ist beispielsweise die Untersuchung topologischer Räume
auf \href{Invariante_(Mathematik)}{Invarianten}. Mittels der Theorie
über \href{Homologie_(Mathematik)}{Homologien} und
\href{Kohomologie}{Kohomologien} sucht man in der algebraischen
Topologie nach solchen Invarianten.

\subsubsection{Geometrische Topologie}\label{geometrische-topologie}

Die geometrische Topologie befasst sich mit zwei-, drei- und
vierdimensionalen \href{Mannigfaltigkeit}{Mannigfaltigkeiten}. Der
Begriff zweidimensionale Mannigfaltigkeit bedeutet das gleiche wie
\href{Fläche_(Mathematik)}{Fläche} und drei- und vierdimensionalen
Mannigfaltigkeiten sind entsprechende Verallgemeinerungen. Im Bereich
der geometrischen Topologie interessiert man sich dafür, wie sich
Mannigfaltigkeiten unter stetigen Transformationen verhalten.

\subsubsection{Winkel, Löcher, Flächen,
Knoten}\label{winkel-luxf6cher-fluxe4chen-knoten}

Typische geometrische Größen wie Winkel, Länge und \url{Krümmung}
variieren unter stetigen Abbildungen. Eine geometrische Quantität, die
nicht variiert und für die man sich daher interessiert, ist die
\href{Geschlecht_(Fläche)}{Anzahl der Löcher} einer Fläche.\footnote{}
Da man sich fast nur mit Mannigfaltigkeiten der Dimension kleiner als
fünf beschäftigt, nennt man dieses Teilgebiet der Topologie auch
niedrigdimensionale Topologie. Außerdem gehört die \url{Knotentheorie}
als Teilaspekt der Theorie dreidimensionaler Mannigfaltigkeiten zur
geometrischen Topologie.\footnote{}

\subsection{Anwendungen}\label{anwendungen}

\subsubsection{Allgemein}\label{allgemein}

Da das Gebiet der Topologie sehr weit gefächert ist, findet man Aspekte
von ihr in fast jedem Teilgebiet der Mathematik. Das Studium der
jeweiligen Topologie bildet daher oft einen integralen Bestandteil einer
tieferen Theorie. Topologische Methoden und Konzepte sind somit aus
weiten Teilen der Mathematik nicht mehr wegzudenken. Es seien hier nun
einige Beispiele angegeben:

\begin{itemize}
\tightlist
\item
  Differentialgeometrie
\item
  Mannigfaltigkeiten''
\end{itemize}

\subsubsection{Differentialgeometrie}\label{differentialgeometrie}

In der \url{Differentialgeometrie} spielt das Studium von
\href{Mannigfaltigkeit}{Mannigfaltigkeiten} eine zentrale Rolle. Bei
diesen handelt es sich um spezielle
\href{Topologischer_Raum}{topologische Räume}, d.~h. Mengen, die eine
gewisse topologische Struktur aufweisen. Oft werden sie auch
topologische Mannigfaltigkeiten genannt. Grundlegende Eigenschaften
werden dann mithilfe topologischer Mittel bewiesen, bevor sie mit
weiteren Strukturen versehen werden und dann eigenständige (und nicht
äquivalente) Unterklassen bilden (z.~B.
\href{differenzierbare_Mannigfaltigkeit}{differenzierbare
Mannigfaltigkeiten}, PL-Mannigfaltigkeiten etc.).

\subsubsection{Beispiel Klassifikation von
Flächen}\label{beispiel-klassifikation-von-fluxe4chen}

\emph{Beispielhaftes verwendetes Ergebnis der geometrischen Topologie:
Klassifikation von Flächen}

Geschlossene Flächen sind spezielle Arten von 2-dimensionalen
Mannigfaltigkeiten. Mithilfe der algebraischen Topologie lässt sich
zeigen, dass jede Fläche aus endlich vielen eingebetteten
\href{Polytop_(Geometrie)}{2-Polytopen} besteht, die miteinander entlang
ihrer Kanten verklebt sind. Dies erlaubt insbesondere eine
\href{Klassifikationssatz_für_2-Mannigfaltigkeiten}{Klassifizierung
aller geschlossener Flächen} in 3 Klassen, weswegen man stets annehmen
kann, dass die geschlossene Fläche in einer „Normalform`` vorliegt.

\subsubsection{Funktionalanalysis}\label{funktionalanalysis}

Die \url{Funktionalanalysis} entstand aus dem Studium von
\href{Funktionenraum}{Funktionenräumen}, welche zunächst Abstraktionen
als \href{Banachraum}{Banach-} und \href{Hilbertraum}{Hilberträume}
erfuhren.

\subsubsection{Topologischer Vektorraum}\label{topologischer-vektorraum}

Heute befasst sich die Funktionalanalysis auch allgemeiner mit
unendlichdimensionalen \href{Topologischer_Vektorraum}{topologischen
Vektorräumen}. Dies sind Vektorräume versehen mit einer Topologie,
sodass die grundlegenden algebraischen Operationen des Vektorraums
stetig („kompatibel`` mit der Topologie) sind. Viele in der
Funktionalanalysis untersuchte Konzepte lassen sich allein auf die
Struktur topologischer Vektorräume zurückführen, als welche sich
insbesondere Hilbert- und Banachräume auffassen lassen, sodass sie als
zentraler Untersuchungsgegenstand der Funktionalanalysis angesehen
werden können.

\subsubsection{Deskriptive Mengenlehre}\label{deskriptive-mengenlehre}

Die \href{deskriptive_Mengenlehre}{deskriptive Mengenlehre} befasst sich
mit gewissen „konstruierbaren`` sowie „wohlgeformten`` Teilmengen
\href{Polnischer_Raum}{polnischer Räume}. Polnische Räume sind spezielle
topologische Räume (ohne weitere Struktur) und viele untersuchte
zentrale Konzepte sind rein topologischer Natur. Diese topologischen
Begriffe stehen in Zusammenhang mit Konzepten der
„\url{Definierbarkeit}`` und „\url{Berechenbarkeit}`` aus der
\href{Mathematische_Logik}{mathematischen Logik}, über welche sich so
mit topologischen Methoden Aussagen machen lassen.

\subsubsection{Harmonische Analysis}\label{harmonische-analysis}

Zentraler Untersuchungsgegenstand der
\href{Harmonische_Analysis}{harmonischen Analysis} sind
\href{lokalkompakte_Gruppe}{lokalkompakte Gruppen}, das sind
\href{Gruppe_(Mathematik)}{Gruppen} versehen mit einer kompatiblen
\href{lokalkompakt}{lokalkompakten} topologischen Struktur. Diese
stellen eine Verallgemeinerung der \href{Lie-Gruppe}{Lie-Gruppen} und
somit von Vorstellungen „kontinuierlicher Symmetrien`` dar.

\subsection{Literatur}\label{literatur}

\subsubsection{Zur Geschichte}\label{zur-geschichte}

\begin{itemize}
\item
\item
  \href{Marie-Luise_Heuser}{Marie-Luise Heuser}: ''Die Anfänge der
\end{itemize}

Topologie in Mathematik und Naturphilosophie.'' In: Stephan Günzel
(Hrsg.): \emph{Topologie: Zur Raumbeschreibung in den Kultur- und
Medienwissenschaften.} Transcript, Bielefeld 2007, S. 183--202.

\subsubsection{Lehrbücher}\label{lehrbuxfccher}

\begin{itemize}
\item
\item
\item
\item
\item
\item
  * 
\item
\item
  * 
\item
\end{itemize}

\subsection{Weblinks}\label{weblinks}

\begin{itemize}
\tightlist
\item
  {[}\url{http://www.mathematik.de/ger/information/landkarte/gebiete/topologie/}
\end{itemize}

topologie.html Artikel zur Topologie{]} auf mathematik.de * In:
\emph{The Mathematical Atlas.}

\begin{itemize}
\tightlist
\item
  {[}\url{http://at.yorku.ca/topology/} ''Topology
\end{itemize}

Atlas''{]} * * \emph{{[}\url{http://www.topologywithouttears.net/}
Topology without tears{]}} von Sidney A. Morris: Buch zur Topologie zum
kostenfreien Download (PDF, englisch)

\begin{itemize}
\tightlist
\item
  {[}\url{http://www.savoir-sans-frontieres.com/JPP/telechargeables/Deutch/}
\end{itemize}

das\_topologikon.htm Comic zur Topologie zum kostenfreien Download (PDF,
\emph{deutsch}){]}

\subsection{Quelle}\label{quelle}

Modifizierter Artikel aus Wikipedia aus der \href{Kategorie:Topologie}{
} als \href{Teilgebiet_der_Mathematik}{Teilgebiet der Mathematik}

\hypertarget{refs}{}

\end{document}
